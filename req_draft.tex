\documentclass[10pt,a4paper]{article}
\usepackage[utf8]{inputenc}
\usepackage[german]{babel}
\usepackage{amsmath}
\usepackage{amsfonts}
\usepackage{amssymb}
\usepackage{color}	
\newcommand{\todo}[1]{\textbf{\textsc{\textcolor{red}{(TODO: #1)}}}}
\parindent 0pt		
\author{Maike Maas, Valerie Poser, Sebastian Becking,\\
Marc Jose, Fabian Kosmale, Heiko Becker}
\title{Visualizing commit comments with chromograms}
\begin{document}
\maketitle
%\renewcommand{\baselinestretch}{1.50}\normalsize
\section{Must-Haves:}
\begin{enumerate}
\item The product's front end is a website working with recent and well known browsers (e.g. Firefox, Google Chrome, latest release of Firefox, Opera, Safari, mobile Versions of them).
\item It supports visualization of the commit comments of git repositories from a given URL.
\item The visualization is displayed in a block and a timeline in separated pictures.
\todo{Waere es nicht besser, sich auf eine Darstellung festzulegen fuer die Must-Haves?}
\item The visualization parameters will be a nonempty subset of the parameters listed below. \todo{Besser eine explizite Angabe der Parameter fuer unser Basisprogramm}
\item The visualization parameter includes the commit message.
\todo{Eigentlich unnoetig, unser Projekt heißt nicht umsonst "Visualizing \textit{commit comments} with chromograms". Siehe Must-Haves 2.}
\item The product is released open source and usable with the version of apache which is supported at release time.
\end{enumerate}
\section{May-Haves:}
\begin{enumerate}
\item The product supports visualizing mercurial and subversion repositories for a nonempty subset of the parameters, which is the same set as used in the Must-Haves.
\item The end-user is able to select which parameters he wants to influence the visualization on the website.
\item The end-user is able to customize the color-mapping for the parameters.
\item The product website is responsive (e.g. by using Bootstrap) and works also with mobile clients.\todo{ Is this context responsive means Bla}
\item The user may upload a folder of a git repository instead of an URL.
\item The visualization is downloadable (e.g. PDF).
\item Share the visualization on instagram.
\item Can run with a basic LAMP environment.
\end{enumerate}
\section{Must-not-haves:}
\begin{enumerate}
\item The product has a desktop client.
\item The client supports version control systems other than mercurial, subversion and git.
\end{enumerate}
\section{Parameters:}
\begin{enumerate}
\item user
\item size of changes
\item only added sth.
\item only deleted sth.
\item time of day of commit
\item date of commit
\item commit message
\end{enumerate}
\end{document}