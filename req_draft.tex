\documentclass[12pt]{scrartcl}
\usepackage[utf8]{inputenc}
\usepackage[german]{babel}
\usepackage{amsmath}
\usepackage{amsfonts}
\usepackage{amssymb}
\usepackage{color}	
\newcommand{\todo}[1]{\textbf{\textsc{\textcolor{red}{(TODO: #1)}}}}
\renewcommand{\baselinestretch}{1.50}\normalsize
\parindent 0pt		
\author{Maike Maas, Valerie Poser, Sebastian Becking,\\
Marc Jose, Fabian Kosmale, Heiko Becker}
\title{Visualizing commit comments with chromograms}
\begin{document}
\maketitle
\section{Must-Haves:}
\begin{enumerate}
\item The product's front end is a website working with recent and well known browsers (e.g. Firefox, Google Chrome, latest release of Firefox, Opera, Safari, mobile Versions of them). It contains a top bar with links to the main site, a documentation or help site and one to the site, where the repositories are visualized. The main site contains a short explanation what the sites purpose is and shows some example pictures. The rendering site uses AJAX to display a progress bar while the backend renders the pictures. With AJAX again it displays them when rendering is finished.
\item It supports visualization of the commit comments of git repositories from a given URL. 
\item The visualization is displayed in an overall block view in one picture.
\item The visualization parameter is the commit message.
\item The product is released open source and usable with the version of apache which is supported at release time.
\item For every rendered image exists a legend which maps the colors to the corresponding strings.
\item The applications language is english.
\end{enumerate}
\section{May-Haves:}
\begin{enumerate}
\item The product supports visualizing mercurial and subversion repositories for commit message and kind of change as explained above.
\item The end-user is able to select which parameters he wants to influence the visualization on the website by using a dropdown menu on the rendering page. In addition to the ones mentioned before we allow paramters from the list below.
\item The end-user is able to customize the color-mapping for the parameters with  a dropdown menu on the rendering page.
\item The product website is responsive (e.g. by using Bootstrap) and works also with mobile clients. In this context responsive means, the layout is optimized for the current browser being used by the enduser.
\item The user may upload a folder of a git repository instead of an URL.
\item The visualization is downloadable (e.g. PDF) on a separate page or on the rendering page.
\item The visualizations can be posted on google+, facebook, twitter and instagram.
\item The application can run with a basic LAMP environment when downloaded by any new end user and has a well documented installation manual.
\item The website can display chromograms for repositories, which are protected by a password. Therefore an additional field to enter the password is displayed on the rendering page.
\item A block of the rendered image can be clicked and then displays its information from the rendering.
\item The application is available in multiple languages.
\item The rendered image is scalable.
\item The application supports thes parameters:
\begin{enumerate}
\item user: 
(the person that made the commit to the repository)
\item kind of change:  
(the numbers of lines that were added, modified or deleted with the commit we're working with)
\item only added sth.:
(taking only the commits into account that added lines)
\item only deleted sth.:
(taking only the commits into account that removed lines)
\item time of day of commit:
(the time of day at which the commit we're currently working with was made)
\item date of commit:
(the date at which the commit we're currently working with was made)
\item commit message:
(the commit message of the user)
\end{enumerate}
\item The parameters may influence a blocks color, size, alignment and order.
\item The application can cache repositories that we're already displayed by any user.
\end{enumerate}
\newpage
\section{Won't-have:}
\begin{enumerate}
\item The product has a desktop client and can be used without a web browser.
\item The client supports version control systems other than mercurial, subversion and git.
\item The software has a native app for any mobile operating system.
\item The application provides a user login interface where users can manage their pictures or their git repositories.
\item The application provides acces to the git repositories.
\item The application provides a visualization of the structure of the repository.
\item The application provides acces to ssh key protected repositories.
\item The application can create movies.
\end{enumerate}
\end{document}