\documentclass[11pt]{scrartcl}
\usepackage{microtype}
\usepackage[utf8]{inputenc}
\usepackage[english]{babel}
\usepackage{amsmath}
\usepackage{amsfonts}
\usepackage{amssymb}
\usepackage{csquotes}
\usepackage{color}	
\usepackage{calc}
\usepackage{enumitem}
\newcommand{\todo}[1]{\textbf{\textsc{\textcolor{red}{(TODO: #1)}}}}
\renewcommand{\baselinestretch}{1.0}\normalsize
\parindent 0pt
\author{Maike Maas \and Valerie Poser \and Sebastian Becking \and
        Marc Jose \and Fabian Kosmale \and Heiko Becker}
\title{Visualizing commit comments with chromograms}
\begin{document}
\maketitle

\section{Use Case}
\begin{description}[leftmargin=!,labelwidth=\widthof{\bfseries Frequency of use:}]
	\item[Use-case:] Visualizing a public Git repository 
	\item[Primary actor:] User of a desktop browser 
	\item[Goal in context:] Visualizing the repository in an overall block view with standard parameters 
	\item[Preconditions:] Browser with a working Internet connection and an URL to a Git repository
	\item[Trigger:] The user wants to get a visualization of his Git repository 

	\item[Scenario:]
		\begin{enumerate}[leftmargin=1.5em]
			\item The user opens the website and arrives at the home page.
			\item The user enters the URL to his Git repository. 
			\item The user clicks on \enquote{Visualize}. (See exception E1, E2, E3, E4, E5)
			\item The website shows a progress bar with an
				informative text displayed beneath
				it. (See exception 6, 7)
			\item The website shows the repository's visualization.
		\end{enumerate}
	\item[Exception Flow:]
		\begin{enumerate}[leftmargin=1.5em]
			\item[E1] The URL is illegal
			\begin{enumerate}
				\item[1] The website shows an error message below the input field: \enquote{You need to enter a legal link to a Git 						repository}.
				\item[] Return to Step 2 of the main scenario.
			\end{enumerate}
			
			\item[E2] The URL does not point to a Git repository. 
				\begin{enumerate}
					\item[1] The website shows an error message below the input field: \enquote{You need to enter a legal 		
						link to a git repository}.
					\item[] Return to Step 2 of the main scenario.
				\end{enumerate}
				
			\item[E3] The Repository is password protected
				\begin{enumerate}
					\item[1] The website shows an
						error message below the input field: \enquote{The Git repository has to be without 
						password protection}.
					\item[] Return to Step 2 of the main scenario.
				\end{enumerate}
				
			\item[E4] The repository is protected by a private SSH key.
				\begin{enumerate}
					\item[1] The website shows an error message below the input field: \enquote{The Git repository 
							has to be without SSH key protection}.
					\item[] Return to Step 2 of the main scenario.
				\end{enumerate}
				
			\item[E5] The repository is empty
				\begin{enumerate}
					\item[1]  The website displays an error message below the input field: \enquote{The repository is empty}.
					\item[] Return to Step 2 of the main scenario.
				\end{enumerate}	
										
			\item[E6] The server hosting the repository does not respond.
				\begin{enumerate}
					\item[1] Return to the home page.
					\item[2] The website shows the homepage with an error message after timeout below the input 	
						field: 	\enquote{The Git repository is currently unavailable. Please try again later}.
					\item[] Return to Step 2 of the main scenario.
					
				\end{enumerate}
				
			\item[E7] The repository is too big.
				\begin{enumerate}
					\item[1] Return to the home page.
					\item[2] The server attempts to obtain a 
						smaller copy of the repository by doing a shallow clone, containing only
						the 50 most recent commits. If the repository is still too big, the
						website shows the homepage with an error message after timeout below the input 	
						field: \enquote{The repository is too big. Please try another one}.
					\item[] Return to Step 2 of the main scenario.
				\end{enumerate}
		\end{enumerate}

	\item[Priority:] high
	\item[Frequency of use:] Very often -- this is the main functionality of the system

\end{description}	
	
\section{Use Case}

\begin{description}[leftmargin=!,labelwidth=\widthof{\bfseries Frequency of use:}]

	\item[Use-case:] Reading the documentation of the project
	\item[Primary actor:] User of a desktop browser 
	\item[Goal in context:] Reading the documentation of the project
	\item[Preconditions:] Browser with a working Internet connection
	\item[Trigger:] The user wants to inform himself about the project
	\item[Scenario:]
		\begin{enumerate}[leftmargin=1.5em]
			\item The user opens the website and arrives at the home page.
			\item The user clicks on the button in the upper right corner labeled with a book.
			\item The website opens the documentation page.
			\item The user reads the documentation and is informed.
			\item The user clicks on \enquote{return}.
			\item The user is back on the home page.
		\end{enumerate}
	
	\item[Priority:] medium
	\item[Frequency of use:] Usually at the first visit of the website
	
\end{description}

\section{Use Case}

\begin{description}[leftmargin=!,labelwidth=\widthof{\bfseries Frequency of use:}]
	\item[Use-case:] Understand how to interact with the website
	\item[Primary actor:] User of a desktop browser 
	\item[Goal in context:] Understand how to interact with the website
	\item[Preconditions:] Browser with a working Internet connection
	\item[Trigger:] The user wants to inform himself how to interact with the website

	\item[Scenario:]
		\begin{enumerate}[leftmargin=1.5em]
			\item The user opens the website and arrives at the home page.
			\item The user clicks on the button \enquote{Help}.
			\item The website overlays a help box with explaining texts and a link to the documentation.
			\item The user reads the helpful information and found a way to interact with the website.
			
		\end{enumerate}

	\item[Priority:] medium
	\item[Frequency of use:] Usually at the first visit of the website

\end{description}

\section {Use Case}

\begin{description}[leftmargin=!,labelwidth=\widthof{\bfseries Frequency of use:}]
	\item[Use-case:] Visualizing an example repository 
	\item[Primary actor:] User of a desktop browser 
	\item[Goal in context:] Visualizing the repository in an overall block view with standard parameters 
	\item[Preconditions:] Browser with a working Internet connection
	\item[Trigger:] The user wants to get a visualization of an example repository

	\item[Scenario:]
		\begin{enumerate}[leftmargin=1.5em]
			\item The user opens the website and arrives at the home page.
			\item The user clicks on \enquote{JQuery}.
			\item The website enters the URL to the repository into the input field.
			\item The user clicks on \enquote{Visualize}.
			\item The website shows a progress bar with an informative text displayed beneath
				it. (See exception E1)
			\item The website shows the repository's visualization.
			\item The user takes a screenshot to save the picture. 
		\end{enumerate}

	\item[Exceptions:]
		\begin{enumerate}[leftmargin=1.5em]
			\item[E1] The server hosting the repository does not respond.
				\begin{enumerate}
					\item[1] Return to the home page.
					\item[2] The website shows the homepage with an error message after timeout below the input 	
						field: 	\enquote{The Git repository is currently unavailable. Please try again later}.
					\item[] Return to Step 2 of the main scenario.
					
				\end{enumerate}
		\end{enumerate}

	\item[Priority:] high
	\item[Frequency of use:] Very often

\end{description}

\end{document}
