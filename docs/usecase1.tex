\documentclass[11pt]{scrartcl}
\usepackage{microtype}
\usepackage[utf8]{inputenc}
\usepackage[english]{babel}
\usepackage{amsmath}
\usepackage{amsfonts}
\usepackage{amssymb}
\usepackage{csquotes}
\usepackage{color}	
\usepackage{calc}
\usepackage{enumitem}
\newcommand{\todo}[1]{\textbf{\textsc{\textcolor{red}{(TODO: #1)}}}}
\renewcommand{\baselinestretch}{1.0}\normalsize
\parindent 0pt
\author{Maike Maas \and Valerie Poser \and Sebastian Becking \and
        Marc Jose \and Fabian Kosmale \and Heiko Becker}
\title{Visualizing commit comments with chromograms}
\begin{document}
\maketitle

\begin{description}[leftmargin=!,labelwidth=\widthof{\bfseries Frequency of use:}]
	\item[Use-case:] Visualizing a public Git repository 
	\item[Primary actor:] User of a desktop browser 
	\item[Goal in context:] Visualizing the repository in an overall block view with standard parameters 
	\item[Preconditions:] Browser with a working Internet connection and an URL to a Git repository
	\item[Trigger:] The user wants to get a visualization of his Git repository 

	\item[Scenario:]
		\begin{enumerate}[leftmargin=1.5em]
			\item The user opens the website and arrives at the home page.
			\item The user enters the URL to his Git repository. 
			\item The user clicks on \enquote{Visualize}. (See exception 1, 2, 3, 4, 5)
			\item The website shows a progress bar with an
				informative text displayed beneath
				it. (See exception 6, 7)
			\item The website shows the repository's visualization.
			\item The user takes a screenshot to save the picture. 
		\end{enumerate}
		
		
	\item[Alternate Flow]
		\begin{enumerate}
			\item[2.1.a.] The user clicks on the button \enquote{Help}.
			\item[2.2.a.] The website overlays a help box with explaining texts and a link to the documentation.
			\item[2.3.a.] The user reads the helpful information and found a way to interact with the website.
			\item[2.4.a.] The user closes the help box.
			\item[] Return to Step 2 of the main scenario.
		
			\item[2.1.b.] The user clicks on the button in the upper right corner labeled with a question mark.
			\item[2.2.b.] The website opens the documentation page.
			\item[2.3.b.] The user reads the documentation and is informed.
			\item[2.4.b.] The user clicks on \enquote{return}.
			\item[] Return to Step 2 of the main scenario.
						
			\item[2.1.c.] The user clicks on \enquote{JQuery}.
			\item[2.2.c.] The website enters the URL into the input field.
			\item[2.3.c.] The user clicks on \enquote{Visualize}.
			\item[] Return to Step 4 of the main scenario.
			
		
		\end{enumerate}

	\item[Exceptions:]
		\begin{enumerate}[leftmargin=1.5em]
			\item The URL is illegal $\rightarrow$ The website shows an error message below the 						input field: \enquote{You need to enter a legal link to a Git repository}. (See use-case 				2)
			\item The URL does not point to a Git repository. $\rightarrow$  The website
				shows an error message below the input field: \enquote{You need to enter a legal 		
				link to a git repository}. (See use-case 3)
			\item The Repository is password protected $\rightarrow$ The website shows an
				error message below the input field: \enquote{The Git repository has to be without 
				password protection}. (See use-case 4)
			\item The repository is protected by a private SSH key $\rightarrow$ The
				website shows an error message below the input field: \enquote{The Git repository 
				has to be without SSH key protection}. (See use-case 5)
			\item The repository is empty$\rightarrow$ The website displays an error
				message below the input field: \enquote{The repository is empty}. (See use-case 6)
			\item The server hosting the repository does not respond $\rightarrow$ The
				website shows the homepage with an error message after timeout below the input 	
				field: 	\enquote{The Git repository is currently unavailable. Please try again later}.  
				(See use-case 7)
			\item The repository is too big $\rightarrow$  The server attempts to obtain a
				smaller copy of the repository by doing a shallow clone, containing only
				the 50 most recent commits. If the repository is still too big, the
				website shows the homepage with an error message after timeout below the input 	
				field: \enquote{The repository is too big. Please try another one}. (See use-case 8)
		\end{enumerate}

	\item[Priority:] high
	\item[Frequency of use:] Very often --- this is the main functionality of the system

\end{description}

\end{document}
