\documentclass[12pt]{scrartcl}
\usepackage[utf8]{inputenc}
\usepackage[german]{babel}
\usepackage{amsmath}
\usepackage{amsfonts}
\usepackage{amssymb}
\usepackage{color}	
\newcommand{\todo}[1]{\textbf{\textsc{\textcolor{red}{(TODO: #1)}}}}
\renewcommand{\baselinestretch}{1.25}\normalsize
\parindent 0pt		
\author{Maike Maas \and Valerie Poser \and Sebastian Becking
\and Marc Jose \and Fabian Kosmale  \and Heiko Becker}
\title{Visualizing commit comments with chromograms}
\begin{document}
\maketitle
\section{Must-Haves:}
\begin{enumerate}
\item The product's front end will be a website that will work with recent and well known browsers (Firefox 25.0, Internet Explorer 11, Google Chrome 31.0, Safari 7.0 and Opera 17.0). 
\item The website will contain a top bar with links to the main page, a documentation page and one to the page where the repositories are visualized. 
\item The home page will contain a short explanation about the site's purpose.
\item The home page will also show some example renderings.
\item The system will support visualization of the commit comments of a provided git repository by representing every commit comment as a colored block.
\item A repository will be provided by the user by entering a URL which refers to the repository that should be visualized.
\item The rendering page will display a progress bar while the system is rendering the picture.
\item The rendering page will show the visualization after the rendering process is finished.
\item The visualization will be displayed in a single picture.
\item The visualization will provide an overall block view of all commit
	comments. Overall means that one image will be displayed for all
	commits, not segmented by commit time or modified files. Block view means
	that the image is a rectangle consisting of multiple rectangles
	representing each commit message.
\item The main visualization parameter will be the commit message.
\item The user will be able to select the number of commit messages to visualize. 
\item The product will be released open source.
\item The application language will be English.
\item The application will detect if the repository's hosting server doesn't answer and will inform the user about that fact by printing an error message.
\item The application will detect if it has no access to the remote repository and will inform the user about that fact by printing an error message.
\item The system will support repositories whose commit comments are encoded in
	ANSI-ASCII. It will skip over Unicode glyphs.
\item The application will run on a Linux server having PHP 5.3 and Apache/2.2.24(Unix) installed.
\end{enumerate}
\section{May-Haves:}
\begin{enumerate}
\item The product may support visualizing commit comments of mercurial and
	subversion repositories analogously to how git repositories are
	visualized.
\item There may be an extra customize page where the user may be able to customize the visualization.
\item The user may be able to customize the color-mapping for the parameters on the customize page.
\item The user may be able to select which parameters he wants to influence the visualization on the customize page. In addition to the visualization parameters mentioned before the system may support parameters from the list below (see May-Have 5).
\item The application may support these parameters for the visualization:
\begin{enumerate}
\item user\\
(the person that made the commit to the repository)
\item kind of change\\  
(the numbers of lines that were added, modified or deleted with the commit we're working with)
\item only added something \\
(taking only the commits into account that added lines or files)
\item only deleted something\\
(taking only the commits into account that removed lines or files)
\item time of day of commit\\
(the time of day at which the commit we're currently working with was made)
\item date of commit\\
(the date at which the commit we're currently working with was made)
\item structure of the repository\\
(branches, merges)
\end{enumerate}
\item The parameters may influence a block's color, size, alignment and order.
\item The visualization may be supported by a legend mapping used colors to the corresponding words.
\item The product website may be responsive and may also work with mobile clients. In this context responsive means, the layout will be optimized for the current browser used by the user.
\item The visualization may be downloadable as a PDF or JPEG.
\item The website may show the URL from a random repository of Github as example. 
\item The user may click on a single block within the visualization to see further information about the specific commit comment represented by that block.
\item The application may be available in multiple languages.
\item The visualization of the repository created on the rendering page may be scalable.
\item The application may cache repositories that have already been displayed by the user.
\end{enumerate}
\section{Won't-have:}
\begin{enumerate}
\item The product will not have a desktop client and cannot be used without a web browser.
\item The client will not support version control systems other than Mercurial, Subversion and Git.
\item The software will not have a native app for any mobile operating system.
\item The user will not be able to upload a folder of a git repository instead of an URL.
\item The application will not provide a user login interface where users can manage their pictures or their repositories.
\item The visualizations will not be be shared on Google+, Facebook, Twitter and Instagram.
\item The application will not provide access to the repositories.
\item The application will not provide access to protected repositories.
\item The application will not create animated renderings.
\end{enumerate}
\end{document}
